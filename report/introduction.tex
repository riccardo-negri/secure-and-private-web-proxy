\chapter{Introduction}\label{ch:sample-chapter}

Web proxy services are essential tools that enable users to access web content that might otherwise be unavailable due to various barriers, such as geographical restrictions and technology limitations. These services work by retrieving the required web content, processing it through a web proxy application, and then presenting it to the user all while being connected to a single website, from an observer perspective. While these services offer significant benefits, they also introduce considerable security and privacy risks. Research conducted by Watanabe et al. \cite{watanabe2020melting}  has exposed the numerous vulnerabilities associated with web proxy services, including man-in-the-middle attacks, potential credential thefts, and session hijacking. These risks largely arise from the necessity of trusting the operators of these services.

To address these security concerns, this project investigates the possibility of running a web proxy service within a Trusted Execution Environment (TEE). Leveraging a TEE can significantly reduce many of the security risks tied to traditional web proxy services. This is because a TEE ensures that even if the service operators are compromised, the integrity and confidentiality of the processed content remains protected.

The main aim of this project is to design and implement a secure web proxy service within an enclave. However, the implementation must carefully balance the limited capacity of the SGX enclave with the performance and usability needs and security requirements. The project also involves a thorough security assessment of the system. Performance will be assessed through an experiment comparing the new web proxy with regular browsing and with two commercial web proxies. Furthermore, load testing will be conducted to see how well the proxy handles different levels of traffic.

